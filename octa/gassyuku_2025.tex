\documentclass[unicode,14pt,aspectratio=169]{beamer}

\usepackage{luatexja}% 日本語したい
\usepackage[haranoaji,no-math,deluxe,expert,nfssonly,match,scale=1.0]{luatexja-preset}
\renewcommand{\kanjifamilydefault}{\gtdefault}% 既定をゴシック体に

\usepackage{amssymb,amsmath,ascmac}

\usepackage{multirow}
\usepackage{lltjext}

\usepackage{siunitx}

\newcommand{\dv}[2][]{\dfrac{\mathrm{d} #1}{\mathrm{d} #2}}
\newcommand{\pv}[2][]{\dfrac{\partial #1}{\partial #2}}
\newcommand{\dd}[1]{\mathrm{d} #1}

\usepackage{bm}

\usepackage{animate}
\usepackage{svg}
\usepackage{xparse}
%%%%%%%%%%%%%%%%%%%%%%%
\graphicspath{{../../figures/},{../figures/}}
%%%%%%%%%%%%%%%%%%%%%%%%%%%
\usepackage{tikz}
\usetikzlibrary{shapes,arrows}
\usetikzlibrary{positioning}
\usetikzlibrary{angles,quotes}
\usetikzlibrary{math, calc}
\usetikzlibrary{decorations.markings,decorations.pathmorphing}
%% define fancy arrow. \tikzfancyarrow[<option>]{<text>}. ex: \tikzfancyarrow[fill=red!5]{hoge}
\tikzset{arrowstyle/.style n args={2}{inner ysep=0.1ex, inner xsep=0.5em, minimum height=2em, draw=#2, fill=black!20, font=\sffamily\bfseries, single arrow, single arrow head extend=0.4em, #1,}}
\NewDocumentCommand{\tikzfancyarrow}{O{fill=black!20} O{none}  m}{
\tikz[baseline=-0.5ex]\node [arrowstyle={#1}{#2}] {#3 \mathstrut};}
\newcommand{\highlight}[2][yellow]{\tikz[baseline=(x.base)]{\node[rectangle,rounded corners,fill=#1!10](x){#2};}}
\newcommand{\highlightcap}[3][yellow]{\tikz[baseline=(x.base)]{\node[rectangle,rounded corners,fill=#1!10](x){#2} node[below of=x, color=#1]{#3};}}

%目次スライド
% \AtBeginSection[]{
%   \frame{\tableofcontents[currentsection]}
% }
%アペンディックスのページ番号除去
\newcommand{\backupbegin}{
\newcounter{framenumberappendix}
\setcounter{framenumberappendix}{\value{framenumber}}
}
\newcommand{\backupend}{
\addtocounter{framenumberappendix}{-\value{framenumber}}
\addtocounter{framenumber}{\value{framenumberappendix}} 
}

%%%%%%%%%%%  theme  %%%%%%%%%%%
% \usetheme{Copenhagen}
\usetheme{metropolis}
% \usetheme{CambridgeUS}
% \usetheme{Berlin}

%%%%%%%%%%%  inner theme  %%%%%%%%%%%
% \useinnertheme{default}

% %%%%%%%%%%%  outer theme  %%%%%%%%%%%
\useoutertheme{default}
% \useoutertheme{infolines}

%%%%%%%%%%%  color theme  %%%%%%%%%%%
%\usecolortheme{structure}

%%%%%%%%%%%  font theme  %%%%%%%%%%%
\usefonttheme{professionalfonts}
%\usefonttheme{default}

%%%%%%%%%%%  degree of transparency  %%%%%%%%%%%
%\setbeamercovered{transparent=30}

%%%%%%%%%%%  numbering  %%%%%%%%%%%
% \setbeamertemplate{numbered}
\setbeamertemplate{navigation symbols}{}
\setbeamertemplate{footline}[frame number]

%%%%%%%%%%%%%%%%%%%%%%%%%%%%%%%%%%%
% \setbeamertemplate{items}[default]
\setbeamertemplate{enumerate item}[default]

% セクション名だけをTOCに出力
\setcounter{tocdepth}{1}



%%%%%%%%%%%%%%%%%%%%%%%%%%%%%%%%%%%
\title
{OCTA合宿 発表資料}
\author[SDL Inc. 佐々木]{佐々木 裕}
\institute[]{ソフトマターデザインラボ合同会社\\hsasaki@softmatters.net}
\date{2025/11/15}

\logo{\begin{tikzpicture}[overlay,remember picture]
    \node[left=1cm] at (current page.333){
        \includegraphics[width=1.8cm]{SoftmatterDesignLab.png}
    };
    \end{tikzpicture}
}
%%%%%%%%%%%%%%%%%%%%%%%%%%%%%%%%%%
\begin{document}

%%%%%%%%%%%%%%%%%%%%%%%%%%%%%%%%%%
\begin{frame}[noframenumbering]\frametitle{}
	\titlepage
\end{frame}

%%%%%%%%%%%%%%%%%%%%%
\section{今日のお話の流れ}
\begin{frame}
    \frametitle{全体の流れ}
        今日のお話は以下の流れで。
        \begin{itemize}
            \item 過去のシミュレーションでの考えと間違いについて
            \item 最近の名工大林研との共同研究
            \item 最近のシミュレーション
        \end{itemize}
\end{frame}

\begin{frame}
    \frametitle{やりたいこと}
    \begin{itemize}
        \item 強靭なネットワークのためのクライテリアを知りたい。
        \begin{itemize}
            \item なぜ壊れやすいものと強いものができるのか?
            \item 強いというのは、単純にモデュラスが高いとは違う。 
            \item 過去のマクロな知見よりヒステリシス挙動が重要
        \end{itemize}
        \item (私の妄想)
        \begin{itemize}
            \item ファントムネットワークの挙動が重要?
            \begin{itemize}
                \item 適正なネットワーク構造⇒ファントムネットワークの振る舞い
                \item ジャンクションポイント近傍の環境が重要
            \end{itemize}
            \item 破壊進展の時間スケール
            \begin{itemize}
                \item 幅広い周波数でのヒステリシス挙動と関係
            \end{itemize}
            \item それらの複合で靭性が支配されるのでは?
        \end{itemize}
    \end{itemize}
\end{frame}

\begin{frame}
    \frametitle{ビトリマーの実験系}

    \begin{itemize}
        \item ビトリマー⇔結合交換性を有するネットワーク
        \item 最近の実験結果
        \begin{itemize}
            \item 適正なビトリマーネットワークを結合交換条件でアニール
            \item 高温でのゴム弾性がきれいに温度依存を示すようになる。
            \item 若干のラバープラトーの低下⇔普通はストランド切断と判断
            % \item ガラス状態でのセグメント移動では、この挙動は説明できない。
        \end{itemize}
        \item 佐々木の憶測
        \begin{itemize}
            \item 結合の繋ぎ変えによりネットワーク構造が整理され、
            \item ジャンクションポイント周りの環境が改善されているのでは?
        \end{itemize}
        \item<2> \alert{(未発表データなので極秘)}
        \begin{itemize}
            \item \alert{中性子散乱において、架橋点密度の空間ゆらぎがきれいになっている可能性が示唆された。}
        \end{itemize}
    \end{itemize}
\end{frame}

\begin{frame}
    \frametitle{最近の私のシミュレーション}

    \begin{itemize}
        \item ストランド長に対応する初期構造を見直し
        \begin{itemize}
            \item 粒子密度はKG鎖に合わせて0.85
            \item ストランドセグメント数と収縮率を変化
            \item 経路長に対応して、Affin⇒Phantomへと遷移
        \end{itemize}
        \item 結合交換性を有するネットワークをシミュレート
        \begin{itemize}
            \item OCTAにある結合切断と結合生成を組み合わせ
            \item 結合交換挙動を表現
        \end{itemize}
        \item 適当なパラメタを設定すれば、それらしい変化が可能。
        \begin{itemize}
            \item ストランド長の分布関数がガウス分布に近づくことが確認できた。
        \end{itemize}
    \end{itemize}
\end{frame}

\section{2023年のレオロジー討論会資料}
\subsection{Adhesive Bonding Technology as a Key to Multi-Materialization}
\begin{frame}
    \frametitle{Adhesive Bonding Technology}
		\begin{columns}[c, onlytextwidth]
			\column{.4\linewidth}
					\centering
						\includegraphics[width=\textwidth]{adhesive_car2.png}

						\vspace{5mm}
						\includegraphics[width=\textwidth]{adhesive_car.png}
				
			\column{.58\linewidth}
			\begin{itemize}
				\item For {Energy conservation}
					\begin{itemize}
						\item weight reduction of cars
						\item \alert{multi-materialization}
						\item \alert{adhesive bonding} technology is a key
					\end{itemize}
				\item durability in long-term use is important
					\begin{itemize}
						\item Especially for alert{fatigue tests}
						\item \alert{reliability of polymer materials is still ambiguous}
					\end{itemize}
			\end{itemize}
			
		\end{columns}
\end{frame}

\begin{frame}
	\frametitle{Mechanical Hysteresis Loss and Fracture Energy}
	\vspace{-1mm}
		% \begin{block}{}
			\begin{columns}[c, onlytextwidth]
				\column{.7\linewidth}
					\begin{itemize}
						\item Mechanical Hysteresis Loss 
							\begin{itemize}
								\item Reduced stress on unloading
								\item Energy dissipation during cycle
								\item \alert{Positive correlation} with fracture energy\footnote{
									\scriptsize{K.A.Grosch, J.A.C.Harwood, A.R.Payne, Rub. Chem. Tech., 41, 1157(1968)}
								}
							\end{itemize}
						% \item \alert{Possitive correration} with fracture energy\footnote{
						% 		\scriptsize{K.A.Grosch, J.A.C.Harwood, A.R.Payne, \\Rub. Chem. Tech., 41, 1157(1968)}
						% 	}
						% 	\begin{itemize}
						% 		\item \alert{変形温度}にも強く依存
						% 		\item SBRのガラス転移温度との距離?
						% 	\end{itemize}
						\item The origin of Hysteresis Loss\footnote{
							\scriptsize{A.R.Payne, J.Poly.Sci.:Sympo., 48, 169(1974)}
						}
						\begin{itemize}
							\item \alert{Viscoelastics}
							\color{blue}
							\item Crystallization
							\item Derived by added filler
						\end{itemize}
					\end{itemize}
				\column{.25\linewidth}
				\begin{center}
					\vspace{-2mm}
					\includegraphics[width=\textwidth]{hysteresis_curve.png}

					% \vspace{5mm}
					\includegraphics[width=\textwidth]{hyst_break2.png}
				\end{center}
			\end{columns}
\end{frame}

\begin{frame}
	\frametitle{Andrews Theory for Rubber Toughness}
			\begin{columns}[c, onlytextwidth]
				\column{.75\linewidth}
				\begin{itemize}
					\item Focused on \alert{stress field around the crack}\footnote{
						\scriptsize
			{E.H.Andrews, Y.Fukahori, J. of Mat. Sci. 12, 1307 (1977)}
					}
						\begin{itemize}
							\item \textcolor{blue}{Stress Loading zone}
							\item \textcolor{red}{Unloading one}
							\item divided by stress maximum line
						\end{itemize}
					\item On the progress of the crack, 
						\begin{itemize}
							\item \textcolor{green}{stress field is transit}
							\item Hysteresis Loss$\Rightarrow${Energy Dissipation}
							\item The progress of Crack is \alert{Suppressed}
						\end{itemize}
					\item Bigger Hysteresis Loss results in  Higher Toughness.
				\end{itemize}
				\column{.25\linewidth}
					\begin{center}
						\includegraphics[width=.85\textwidth]{crack.png}
					\end{center}
			\end{columns}
\end{frame}

\subsection{Theoretical Models for Rubber}
\begin{frame}
    \frametitle{Classical Theory of Rubber Elasticity}
        \vspace{-3mm}
		\begin{columns}[c, onlytextwidth]
			\column{.48\linewidth}
				% \begin{exampleblock}{Neo-Hookean Model}
					Neo-Hookean Model
					% 第1不変量のみを対象
						\scriptsize
						\begin{align*}
							&W = C_1 (I_1-3) \\
							&\text{against Uniaxial elongation} \\
							&\sigma_{nom} = 2 C_1\left(\lambda - \dfrac{1}{\lambda^2}\right) = G \left(\lambda - \dfrac{1}{\lambda^2}\right)
						\end{align*}
				% \end{exampleblock}
			\column{.48\linewidth}
				% \begin{alertblock}{Mooney-Rivlin Model}
					Mooney-Rivlin Model
					% 高次の項をおとす
					\scriptsize
					\begin{align*}
						&W = C_1 (I_1-3) + C_2(I_2-3) \\
						&\text{against Uniaxial elongation} \\
						&\sigma_{nom} = 2 \left(C_1 + C_2\dfrac{1}{\lambda} \right) \left(\lambda - \dfrac{1}{\lambda^2}\right)
					\end{align*}
				% \end{alertblock}
		\end{columns}
		\vspace{3mm}
		% \begin{block}{With or without Junction Poinits fluctuation}
		\uncover<2>{With or without Junction Poinits fluctuation
            \vspace{3mm}
			\begin{columns}[c, onlytextwidth]
				\column{.48\linewidth}
				\small
				\color{blue}{Affine Network Model
				\footnote{
					\tiny{P.J. Flory, Principles of Polymer Chemistry, (1953)}
				}
				}
				\vspace{-2mm}
				\scriptsize
				\begin{align*}
					% &\text{Affine Network Model}\\
					&G_{affine} = \nu k_B T  \\
					&\text{$\nu$: Number density of strands in the system}
				\end{align*}
				\column{.48\linewidth}
				\small
				\color{magenta}{Phantom Network Model
				\footnote{
					\tiny{H.M. James, E.J. Guth, Chem. Phys., 21, 6, 1039 (1953)}
				}
				}
				\vspace{-2mm}
				\scriptsize
				\begin{align*}
					% &\text{Phantom Network Model}\\
					&G_{phantom} = \nu k_B T \left(1 - \dfrac{2}{f}\right) \\
					&\text{$f$: Functionality of Junction Points}
				\end{align*}
				\normalsize
			\end{columns}}
		% \end{block}
\end{frame}

\begin{frame}
	\frametitle{Constraint Factors for Junction Points and Strands}
		\vspace{-2mm}
		\begin{alertblock}{Vicinity of Junction Point}
			\begin{columns}[c, onlytextwidth]
				\column{.78\textwidth}
				\vspace{-3mm}
				\begin{itemize}
					\item Surrounded by many of \alert{adjacent strands.}
					\item Fluctuation of junctions are \alert{suppressed}. 
				\end{itemize}
				\column{.2\textwidth}
				\centering
				\includegraphics[width=.9\textwidth]{JP_vicinity.png}
			\end{columns}
		\end{alertblock}
		\vspace{-1mm}
		\begin{block}{Effect of other strands (Combination of $G_c$ and $G_e$)}
			\only<1>{
				\begin{itemize}
					\item Suppress the fluctuation of Junction Point
					\begin{itemize}
						\item Deviate from Phantom Network Model and higher $G_c$
					\end{itemize}
					\item Strands Entangles each other
					\begin{itemize}
						\item Works as a Junction Point
						\item Generate additional $G_e$
					\end{itemize}
				\end{itemize}
				Storage modulus $G$ is \alert{combination of $G_c$ and $G_e$}
			}
			\only<2>{
				\begin{columns}[c, onlytextwidth]
                    \column{.1\textwidth}
					\column{.6\textwidth}
					\begin{itemize}
						\item Constrained Junction Model
						\begin{itemize}
							\item  $G$ approaches to $G_c$.\footnote{\tiny{P.J.Flory, J.Chem.Phys., 66, 12, 5720 (1977)}}
						\end{itemize}
						\item Topological relationships
						\begin{itemize}
							\item Contribution of entanglement.\footnote{\tiny{D.S.Pearson and W.Graessley, Macromol., 11, 3, 528 (1978)}}
							\vspace{-2mm}
							\scriptsize
							\begin{align*}
								G_e = T_e G_N^0
							\end{align*}
						\end{itemize}
					\end{itemize}
					\column{.2\textwidth}
					% \vspace{-2mm}
					\includegraphics[width=\textwidth]{Constrained_Juntion.pdf}
					\vspace{1mm}
					\includegraphics[width=\textwidth]{topological_effect_ring.png}
                    \column{.1\textwidth}
				\end{columns}
			}
		\end{block}
\end{frame}

\setcounter{footnote}{0}
\begin{frame}
	\frametitle{Recent approach for Constraints (Entanglements)}
	% \vspace{-2mm}
		\begin{itemize}
			\item Diffused-Constraint Model
			\begin{itemize}
				\item Confining potential affect all points along the chain.\footnote{\tiny{A. Kloczkowski, J.E. Mark, B. Erman, Macromol., 28, 5089 (1995)}}
			\end{itemize}
			\item Nonaffine Tube Model
			\begin{itemize}
				\item Improved model of "Edwards' Tube Model".\footnote{\tiny{M. Rubinstein, S. Panyukov, Macromol., 30, 25, 8036 (1997)}}
			\end{itemize}
			\item \alert<2>{Slip-tube Model}
			\begin{itemize}
				\item A pairwise interaction of chains is introduced.\footnote{\tiny{M. Rubinstein, S. Panyukov, Macromol., 35, 6670 (2002)}}
			\end{itemize}
		\end{itemize}

		\vspace{1mm}
		\centering
		\visible<2>{
			\includegraphics[width=.75\textwidth]{rubinstein_4.png}
		}
\end{frame}

\subsection{Objectives}
\setcounter{footnote}{0}
\begin{frame}
	\frametitle{Random Networks as a key for PNM}
		\begin{columns}[c, onlytextwidth]
			\column{.72\linewidth}
			\begin{itemize}
				\item Introduction of \alert{Random Connectivity}.
					\item \alert{Criteria for PNM} is fulfilled\footnote{
								\scriptsize{P. J. Flory, Proc. R. Soc. London. A, 351, 351 (1976)}
						}.
					\begin{itemize}
						\item the mean values $\bar{\bm{r}}$ of strands are \alert{fluctuate}
						\item fluctuations $\Delta \bm{r} = \bm{r} - \bar{\bm{r}}$ are \alert{Gaussian}
						\item the mean-square fluctuations \alert{depend only on structure}
					\end{itemize}
					% \end{itemize}
				\item Previous Work for Random Network
				\begin{itemize}
					\item \alert{Random endcrosslink for telechelics}\footnote{
						\scriptsize{G.S. Grest, et.al., Non-Cryst. Solids, 274, 139 (2000)}
						}
					\item \alert{Primitive Chain Network Simulation}\footnote{
						\scriptsize{Y. Masubuchi, Nihon Reoroji Gakkaishi, 49, 2, 73 (2021)}
					}
				\end{itemize}
			\end{itemize}
			\column{.26\linewidth}
				\centering
					\includegraphics[width=\textwidth]{random_NW.png}
		\end{columns}
\end{frame}

\begin{frame}
	\frametitle{Objectives}

	\begin{itemize}
		\item Recent approach for rubber elasticity models are based on Phantom Network Model.
		\item Introducing random connectivity, MD simulation studies were carried out.
		\item To investigate the criteria for Phantom Network Model, Two model chains are used.
		\begin{enumerate}
			\item Employing phantom chain, basics for PNM is examined.
			\item Changing the chain to KG Chain, constraints effects are investigated.
			\begin{itemize}
				\item Excluded Volume Effect
				\item No mutual crossing of Strands
			\end{itemize}
		\end{enumerate}
	\end{itemize}
\end{frame}


\section{Simulation}
\subsection{Generation Recipe of Random Networks}

\begin{frame}
	\frametitle{Generation of Initial Structure of Random Networks}
	\begin{enumerate}
		\item 8-Chain Model is used as starting structure in \alert{Real space}.
			\begin{itemize}
				\item Randomly selected edge is removed until desired functionality.
				% \item Topological model is generated. 
			\end{itemize}
		\item Randomness is introduced in \alert{topological space}.
			\begin{itemize}
				\item By \alert{edge exchange}, random connectivity is introduced.
                %  for each node.
			\end{itemize}	
		\item Corresponding real space structure is generated.
            \begin{itemize}
                \item According to e2e distance of strand, system size is set.
            \end{itemize}
	\end{enumerate}

	\vspace{-1mm}
	\begin{columns}[T, onlytextwidth]
		\column{.3\linewidth}
			\includegraphics[width=\textwidth]{8_per.png}
		\column{.3\linewidth}
			\vspace{-5mm}
			\begin{center}
				\includegraphics[width=.6\textwidth]{Network.png}
			\end{center}
		\column{.3\linewidth}
			\includegraphics[width=\textwidth]{bond_exchg.png}
	\end{columns}
\end{frame}

\subsection{Phantom and KG Chains as Strands}
\begin{frame}
	\frametitle{Phantom and KG Chains as Strands}
	\begin{itemize}
		\item Phantom Chain:
		\begin{itemize}
			\item No Excluded Volume is set (no segmental interaction).
			\item "Force Cap LJ" is set as Angle Potential to enumerate e2e length of KG Chains.
			\item Harmonic bond(k=1000)
		\end{itemize}
		\item KG Chain:
		\begin{itemize}
			\item Excluded Volume is set by Repulsive LJ Potential.
			\item Bond Potential is set to FENE.
			\item Because of above two potentials, No Chain crossing will occur.
		\end{itemize}
	\end{itemize}
\end{frame}

\subsection{Simulation Conditions}
\begin{frame}
    \frametitle{Reluxation of Initial Structure in KG Network}
        % \vspace{-2mm}
			\begin{itemize}
                \item KG Network: KG chain as strand
                \begin{itemize}
                    \item \alert{Relaxation of initial structure} is important.
                \end{itemize}
			\end{itemize}

        
		\begin{columns}[t, onlytextwidth]
			\column{.6\linewidth}
				% \begin{exampleblock}{Initial Structure Relaxation}
					\begin{itemize}
						\item Initial Structure Relaxation
						\begin{itemize}
                            \item According method of Auhl\footnote{
                                \scriptsize{R. Auhl et al. J. of Chem. Phys., 119, 12718 (2003)}
                            }
                            \begin{itemize}
                                \item Using force-capped-LJ pot.
                                \item relaxed by Slow Push Off
                            \end{itemize}
                        \end{itemize}
					\end{itemize}
					\scriptsize
					\begin{align*}
						&U_{FCLJ}(r) = 
						\begin{cases}
						(r-r_{fc})*U_{LJ}^{\prime}(r_{fc}) + U_{LJ}(r_{fc}) \; &r< r_{fc} \\
						U_{LJ}   \;\;\;\;\;\;\; &r \geq r_{fc}
						\end{cases} 
					\end{align*}
				% \end{exampleblock}
                \normalsize
			\column{.28\linewidth}
				\begin{center}
					\includegraphics[width=.9\textwidth]{Ev_fcLJ.png}
				\end{center}
				\vspace{-3mm}
				\scriptsize
				\begin{itemize}
					\item force-capped-LJ Pot.
					\item gradually entangled
				\end{itemize}
            \column{.1\linewidth}
		\end{columns}
\end{frame}

\section{Results}
\subsection{Networks with Phantom Chains}
\begin{frame}
	\frametitle{
		Phantom Chain NW (f=4, N=36)
	}
	\begin{itemize}
		\item Strand-length is set to Equilibrated length (N=36)
		\item Multiplicity is 3 to keep $\rho = 0.85$
	\end{itemize}

	\vspace{3mm}
    \begin{center}
    \begin{minipage}{.7\textwidth}
        \begin{columns}[c, onlytextwidth]
            \column{.4\linewidth}
            \centering
            \includegraphics[width=.65\textwidth]{N36_f4.png}
            \column{.55\linewidth}
            \centering
            \includegraphics[width=.8\textwidth]{contour.png}
        \end{columns}
    \end{minipage}
    \end{center}
	
\end{frame}


\begin{frame}
	\frametitle{
		Strand length Effect for Phantom Chain NW
	}
	\begin{itemize}
		\item Strand-length is varied from 36 to 48 for f=4
		\item System size is \alert{reduced to keep $\rho = 0.85$}
	\end{itemize}
    \begin{center}
    \begin{minipage}{.7\textwidth}
        \begin{figure}[htb]
		\centering
			\includegraphics[width=.8\textwidth]{N36_N40_N48.png}

			\scriptsize{Strand Length Comparison for Shear Stress}
	\end{figure}
    \end{minipage}
    \end{center}
\end{frame}

\begin{frame}
	\frametitle{Comparison of Functionality (f = 3, 4, 6)}
    \begin{center}
    \begin{minipage}{.8\textwidth}
        \begin{block}{For Varied Functionalities}
		\begin{itemize}
			\item With proper strand length
			\item Modulus changed according to theory
		\end{itemize}

		\vspace{5mm}
		\centering
			\includegraphics[width=\textwidth]{compare_346.png}
	\end{block}
    \end{minipage}
    \end{center}
\end{frame}

\subsection{KG Chain Networks}
\begin{frame}
	\frametitle{Mechanical Responce for KG Chains(f=4, N=48)}
	\vspace{-3mm}
	\begin{alertblock}{4-Chain Random Network with KG Chain}
		\begin{itemize}
			\item Excluded Volume Effect by non-bonding LJ Potential.
			\item No strands mutual crossing by FENE bond.
		\end{itemize}
	\end{alertblock}
	\vspace{-4mm}
    \begin{center}
    \begin{minipage}{.8\textwidth}
        \begin{columns}[T, onlytextwidth]
		\column{.48\linewidth}
			\begin{block}{Phantom Chain}
				\includegraphics[width=\textwidth]{4chain_N50_PNM_shear.png}
			\end{block}
			
		\column{.48\linewidth}
			\begin{block}{KG Chain}
				\includegraphics[width=\textwidth]{4chain_N50_shear.png}
			\end{block}
	\end{columns}
    \end{minipage}
    \end{center}
\end{frame}

\begin{frame}
    \frametitle{Analysis of Entanglements in Network: Z1-code}
        % \begin{center}
        % \begin{minipage}{.9\textwidth}
            \begin{columns}[c, onlytextwidth]
            \column{.25\linewidth}
                \includegraphics[width=\textwidth]{z_cord_4Chain.png}
				% \caption{Entanglements by Z-1 Code}
				
            \column{.75\linewidth}
            \begin{block}{Comparison with Homopolymer Melt}
                \begin{itemize}
                    \item Z is number of entanglements per chain
                    % \item 今回のネットワークは、\\ホモポリマーと同等
                \end{itemize}
                \scriptsize
                \begin{center}
                    \begin{tabular}{c||c|c} \hline
                        &Homo & 4 Chain NW \\ \hline \hline
                        Segments& 50& 48 \\ \hline
                        Chains & 200& 768 \\ \hline
                        Entanglements& 204& 800\\ \hline
                        Entangled Chains&134&557 \\ \hline
                        \alert{$<Z>_{Z1}$}&\alert{1.02}& \alert{1.04}\\ \hline
                    \end{tabular}
                \end{center}
            \end{block}
        \end{columns}
        
        \begin{itemize}
            \item \alert{Z1-code?}
            \begin{itemize}
                \item an algorism to visualize and count entanglements\footnote{
                M. Kröger, Comput. Phys. Commun. 168, 209 (2005)
                }
            \end{itemize}
        \end{itemize}
\end{frame}

\begin{frame}
    \frametitle{Reduced Entanglements by NPT Model}
            \begin{columns}[c, onlytextwidth]
            \column{.3\linewidth}
                \footnotesize
				\begin{itemize}
					\item 4-Chain-\alert{NPT}\\
					$<Z>_{Z1}$ = 0.36

					\includegraphics[width=.55\textwidth]{z_cord_NPT_4Chain.png}
					\item 4-Chain-NVT\\
					$<Z>_{Z1}$ = 1,04

					\includegraphics[width=.55\textwidth]{z_cord_4Chain.png}
				\end{itemize}
			\column{.6\linewidth}
			\begin{block}{$G(t)$}
				\begin{itemize}
					\item Step Deformation($\lambda=2.0$)
					\item Reduced Modulus
					% \item<2> \textcolor{blue}{定数を足せばKGと類似}
				\end{itemize}
					\includegraphics[width=.7\textwidth]{gt_4chain_comp.png}
					% \includegraphics<2>[width=\textwidth]{gt_NPT_mod.png}
				\end{block}
		\end{columns}
        % \end{minipage}
        % \end{center}
\end{frame}

\begin{frame}
	\frametitle{Entanglement effect in Slip-tube Model}
			Theoretical model by Rubinstein\footnote{
				\scriptsize{M. Rubinstein, S. Panyukov, Macromolecules, 35, 6670 (2002)}
				}
			\vspace{-3mm}
			\scriptsize
			\begin{align*}
				G_c = \nu k_B T \left(1-\dfrac{2}{\phi} \right), \quad G_e = \dfrac{4}{7} \nu k_B T L \\
				% &\text{where $\nu$ is the number density of network chains,} \\
				\text{and L is the number of slip-links per network chain}
			\end{align*}
	\centering
			\includegraphics[width=.7\textwidth]{Entanglement_Comp.png}
\end{frame}

\subsection{Relaxation in KG Networks}

\begin{frame}
	\frametitle{4-Cain NW のせん断変形時のヒステリシス}
	\begin{itemize}
		% \item 変形速度の低減により、$\gamma<1$ 程度の小さなひずみでは Phantom Network Model:PNM に漸近
		\item PNM へと漸近する変形速度 ($\dot{\gamma} = 2e^{-4}$) で複数回の連続した変形に対しても迅速な回復を伴った力学的ヒステリシス (Hysteresis loss $\simeq$ 0.34) を示した。
	\end{itemize}

	\begin{columns}[totalwidth=\linewidth]
		\column{.5\textwidth}
			\centering
				\includegraphics[width=\textwidth]{Shear_Random_4chain_N20.png}
				Stress-Strain Curves for 4-chain NW 
		\column{.5\textwidth}
			\centering
				\includegraphics[width=\textwidth]{CyclicDeform_4chain_rate_2e-4.png}
				Hysteresis Response with Cyclic Deformations
		\end{columns}
\end{frame}


\begin{frame}
	\frametitle{各種の変形条件での力学的ヒステリシス}
		\centering
			\includegraphics[width=\textwidth]{hyst_shear_all.png}
			Hysteresis losses for valid shear rate and maximum deformation
\end{frame}

\begin{frame}
	\frametitle{ヒステリシスロス}
	\begin{itemize}
		\item 変形速度の低下に伴いヒステリシスロスは減少
		\item \textcolor{red}{$\dot{\gamma} \sim 1e^{-5}$ 程度のオーダーの時間スケールで消失}
	\end{itemize}
			\centering
				\includegraphics[width=.6\textwidth]{hyst_shear.png}\\
					Comparison of Hysteresis losses for 4-Chain NW
\end{frame}

\begin{frame}
	\frametitle{Conclutions}

	\begin{itemize}
		\item Introducing random connectivity, MD simulation studies were carried out.
		\item To investigate the criteria for Phantom Network Model, Two model chains are used.
		\begin{itemize}
			\item Employing phantom chain, basics for PNM is examined.
			\begin{itemize}
				\item Proper strand length is the key for PNM.
				\item Functionality effect was confirmed.
			\end{itemize}
			\item Changing the chain to KG Chain, constraints effects are investigated.
			\begin{itemize}
				\item Trapped Entanglement was explained by Slip-tube Model
				\item Hysteresis 
			\end{itemize}
		\end{itemize}
	\end{itemize}

\end{frame}

\section{ビトリマーの実験系}

\begin{frame}
    \frametitle{ビトリマーの実験系}

    \begin{itemize}
        \item ビトリマー⇔結合交換性を有するネットワーク
        \item 最近の実験結果
        \begin{itemize}
            \item 適正なビトリマーネットワークを結合交換条件でアニール
            \item 高温でのゴム弾性がきれいに温度依存を示すようになる。
            \item 若干のラバープラトーの低下⇔普通はストランド切断と判断
            % \item ガラス状態でのセグメント移動では、この挙動は説明できない。
        \end{itemize}
        \item 佐々木の憶測
        \begin{itemize}
            \item 結合の繋ぎ変えによりネットワーク構造が整理され、
            \item ジャンクションポイント周りの環境が改善されているのでは?
        \end{itemize}
        \item<2> \alert{(未発表データなので極秘)}
        \begin{itemize}
            \item \alert{中性子散乱において、架橋点密度の空間ゆらぎがきれいになっている可能性が示唆された。}
        \end{itemize}
    \end{itemize}
\end{frame}

\section{最近の私のシミュレーション}
\begin{frame}
    \frametitle{最近の私のシミュレーション}

    \begin{itemize}
        \item ストランド長に対応する初期構造を見直し
        \begin{itemize}
            \item ファントム鎖を中心に検討
            \item 粒子密度はKG鎖に合わせて0.85
            \item ストランドセグメント数と収縮率を変化
            \item 経路長に対応して、Affin⇒Phantomへと遷移
        \end{itemize}
        \item 結合交換性を有するネットワークをシミュレート
        \begin{itemize}
            \item OCTAにある結合切断と結合生成を組み合わせ
            \item 結合交換挙動を表現
        \end{itemize}
        \item 適当なパラメタを設定すれば、それらしい変化が可能。
        \begin{itemize}
            \item ストランド長の分布関数がガウス分布に近づくことが確認
        \end{itemize}
    \end{itemize}
\end{frame}

\begin{frame}
    \frametitle{結合交換の設定}
        \begin{columns}[c, onlytextwidth]
            \column{.4\linewidth}
                \begin{itemize}
                    \item 結合乖離条件
                    \item 結合生成条件
                \end{itemize}
            \column{.6\linewidth}
                \centering
            \includegraphics[width=\textwidth]{exchange_gourmet.png}
        \end{columns}
        
\end{frame}

\begin{frame}
    \frametitle{結合交換の効果}
    Random, 4-Chains, bond=FENE, Strand:N=32 \\\alert{only Equilibrated}
    \begin{columns}[c, onlytextwidth]
        \column{.4\linewidth}
            \centering
                \includegraphics[width=\textwidth]{CN_ave_Random_N32_eqn_2.png}

                segmental contour length
        \column{.4\linewidth}
            \centering
            \includegraphics[width=\textwidth]{R_Random_N32_eq2.png}

            R distribution
    \end{columns}
\end{frame}

\begin{frame}
    \frametitle{結合交換の効果}
    Random, 4-Chains, bond=FENE, Strand:N=32 \\\alert{after exchange}
    \begin{columns}[c, onlytextwidth]
        \column{.4\linewidth}
            \centering
                \includegraphics[width=\textwidth]{CN_ave_Random_N32_exc_2.png}

                segmental contour length
        \column{.4\linewidth}
            \centering
            \includegraphics[width=\textwidth]{R_Random_N32_exc_2.png}

            R distribution
    \end{columns}
\end{frame}

\begin{frame}
	\frametitle{ヒステリシスでの違い}

	\begin{columns}[c, onlytextwidth]
		\column{.48\linewidth}
				\centering
					\includegraphics[width=\textwidth]{cycle-5e-4-before.png}
			
					before
		\column{.48\linewidth}
			\centering
					\includegraphics[width=\textwidth]{cycle-5e-4-after.png}

					after
	\end{columns}

\end{frame}
\end{document}
